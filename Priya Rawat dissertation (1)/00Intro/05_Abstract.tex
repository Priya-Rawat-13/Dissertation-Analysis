% TeX_root = ../main.tex
\begin{center}
  {\LARGE{\bf{ABSTRACT}}}
\end{center}

\vspace{2em}

% I don't think you'll need this line and fancy stuff. It looks too
% structured. Take out if you don't want that. 
% \begin{flushleft}
%   \hrule 
%   \vspace{1em}
% 		  Name of the student:	 \textbf{[Full Name]}	\hfill Roll No: \textbf{[Roll No.]} \\
% 		Degree for which submitted:  \textbf{M.Sc} \hfill Department:  \textbf{School of Mathematics} \\
% 		Thesis title: \textbf{[Project Title]}\\
% 		Thesis supervisor:  \textbf{[Project Supervisor]}\\
% 		Date of thesis submission: \textbf{{ \today}\\}
%   \vspace{1em}
% 	\hrule
% \end{flushleft}

%\textit{If you have to structure it as objective, methods, results, and conclusions, do it that way.}


\vspace{2em}


\hspace{1cm}Effective healthcare waste disposal is crucial for maintaining public health and environmental sustainability. This dissertation addresses the challenge of selecting the optimum healthcare waste disposal method by incorporating the TOPSIS (Technique for Order Preference by Similarity to Ideal Solution) method within an intuitionistic fuzzy environment. Furthermore, Aczel-alsina operations are employed to handle the uncertainty and imprecision inherent in healthcare waste management decision-making.

\hspace{1.5cm}The proposed methodology integrates the TOPSIS technique with intuitionistic fuzzy sets to capture the hesitancy and ambiguity associated with healthcare waste disposal alternatives. By considering both the membership and non-membership degrees of alternatives, the decision-making process becomes more robust and comprehensive.

\hspace{1.5cm}To determine the weights of evaluation criteria, an entropy measure is employed, allowing for an objective assessment of the relative importance of each criterion. This helps to avoid biases and subjectivity in the decision-making process, ensuring a more rational and transparent approach.

\hspace{1.5cm}The Aczel-Alsina operations enable the aggregation of intuitionistic fuzzy information and linguistic terms, ensuring a more accurate representation of decision-makers' preferences.

\hspace{1.5cm}Through the application of the proposed methodology to a case study in healthcare waste disposal, the dissertation demonstrates its effectiveness in identifying the most suitable waste disposal method. The objective is to determine the most suitable alternative among four disposal methods: incineration, steam sterilization, microwave, and landfill disposal. The evaluation is performed based on six criteria: cost, waste residuals, release with health effects, reliability, treatment effectiveness, and public acceptance.


\hspace{1.5cm}The results indicate that the optimum alternative is landfill disposal. This finding suggests that considering the given criteria, landfill disposal demonstrates the highest suitability among the evaluated methods.It also indicate that the integration of TOPSIS with intuitionistic fuzzy sets, Aczel-alsina operations, and entropy-based weighting provides a robust framework for healthcare waste disposal decision-making. The methodology assists decision-makers in making informed choices by considering both quantitative and qualitative aspects, thereby contributing to the optimization of healthcare waste management practices.

