% TeX root = ../main.tex
\chapter{Introduction}

 
  \label{sec:start}
 
Healthcare waste management(HCWM) is a critical and complex task that requires careful consideration to protect public health, minimize environmental impacts, and promote sustainable waste management practices. The selection of an optimal healthcare waste disposal method is a crucial decision-making process that involves multiple criteria, subjective preferences, and uncertainty. In this dissertation, we address the challenge of choosing the optimum healthcare waste disposal method using the Technique for Order of Preference by Similarity to Ideal Solution (TOPSIS) method in an intuitionistic fuzzy environment. Additionally, we incorporate linguistic terms, involve multiple decision-makers, utilize the Aczel-Alsina operations, and calculate criteria weights using the entropy measure. The objective is to develop a comprehensive methodology that enables informed and optimal decision-making in healthcare waste management.

\vspace{5mm}

The healthcare sector generates a significant amount of waste, including medical, infectious, and hazardous materials, which pose potential risks to human health and the environment. Inadequate disposal practices can lead to the transmission of diseases, environmental contamination, and adverse public health effects. Therefore, it is crucial to employ appropriate waste management strategies to minimize these risks. The selection of an optimal disposal method requires considering multiple criteria, such as cost, waste residuals, release with health effects, reliability, treatment effectiveness, and public acceptance. These criteria involve subjective judgments and uncertainties, which necessitate the use of decision-making techniques that can handle imprecise information effectively.

\vspace{5mm}

In classical multiple criteria decision-making (MCDM) methods, the ratings and the weights of the
criteria are assumed to be known precisely. In general, crisp data
are inadequate to model real-life situations including HCWM that
incorporate imprecision and vagueness. Moreover, when a large
number of performance attributes are to be considered in the
evaluation process, structuring them in a multi-level hierarchy is
preferred to carry out the analysis more effectively.In an increasingly socioeconomic environment, it is very
difficult for only one DM to address all the important
aspects of a problem. Therefore, decisions are usually
taken by a group of people.

\vspace{5mm}

 The decision-making process for selecting the best disposal technology is often complex and uncertain, involving multiple criteria and conflicting preferences. To capture the imprecise and subjective nature of decision-making in healthcare waste management, we employ intuitionistic fuzzy sets (IFS). IFS extend classical fuzzy sets by introducing a hesitation degree, which represents the degree of uncertainty or lack of knowledge in assigning membership values to elements. By utilizing IFS, we can model the linguistic terms and subjective preferences provided by decision-makers more accurately. This enables a more comprehensive representation of decision-makers' judgments, considering their uncertainties and linguistic expressions.

\vspace{5mm}

In the proposed methodology, we involve multiple decision-makers who possess expertise and knowledge in healthcare waste management. Each decision-maker brings a unique perspective and set of priorities when evaluating the alternative disposal methods. The involvement of multiple decision-makers aims to capture a comprehensive range of preferences and perspectives, ensuring a more robust and reliable decision-making process. The decision-makers' preferences are expressed in linguistic terms, which are effectively modeled using IFS. By considering multiple decision-makers, we account for the subjectivity and diversity of opinions in healthcare waste management.

\vspace{5mm}

To determine the optimal alternative, we employ the TOPSIS method, a widely used multi-criteria decision analysis technique. Originally developed by Hwang and Yoon (1981)\cite{5}, TOPSIS ranks the alternatives based on their similarity to the positive ideal solution and their distance from the negative ideal solution. By applying TOPSIS in an intuitionistic fuzzy environment, we can accommodate the linguistic terms and uncertainty inherent in the healthcare waste disposal decision problem. This approach allows decision-makers to make informed choices based on their linguistic preferences and the defined criteria.

\vspace{5mm}

To aggregate decision-makers' preferences and calculate the overall rankings, we utilize the Aczel-Alsina operations. These operations provide a framework for combining intuitionistic fuzzy numbers, enabling the aggregation of linguistic terms and uncertainties. The Aczel-Alsina operations effectively capture decision-makers' linguistic expressions and uncertainties, facilitating a more accurate representation of their preferences.

\vspace{5mm}

The criteria weights play a crucial role in the decision-making process, as they reflect the relative importance of each criterion. In this dissertation, we calculate the weights of the criteria using the entropy measure. The entropy measure quantifies the level of diversity or uncertainty associated with the criteria values. By utilizing the entropy measure, we can determine the relative importance and contribution of each criterion, considering the available information.

\vspace{5mm}

In the context of healthcare waste management, the optimal alternative among the considered disposal methods has significant implications for public health, environmental sustainability, and resource allocation. In our study, we consider four alternative disposal methods: A1: incineration, A2: steam sterilization, A3: microwave, and A4: landfill disposal. These methods are assessed based on the aforementioned six criteria: cost, waste residuals, release with health effects, reliability, treatment effectiveness, and public acceptance.

\vspace{5mm}

The key objective of this dissertation is to propose a comprehensive methodology that combines the TOPSIS method, intuitionistic fuzzy sets, Aczel-Alsina operations, and entropy measure for optimal healthcare waste disposal method selection. By incorporating linguistic terms, involving multiple decision-makers, and employing advanced decision-making techniques, we aim to enhance the accuracy, robustness, and effectiveness of healthcare waste management decision-making.

\vspace{5mm}

The subsequent chapters of this dissertation will provide detailed explanations of the proposed methodology, including the theoretical foundations, mathematical formulations, and implementation procedures. The rest of the dissertation is organized as follows:

\vspace{5mm}

Chapter 2 provides a comprehensive review of the existing literature and previously published works related to healthcare waste management, decision-making approaches, intuitionistic fuzzy sets, Aczel-Alsina operations, and the entropy weight measure. This review establishes the theoretical and conceptual foundations of our research and highlights the current state of knowledge in the field.

\vspace{5mm}

In Chapter 3, we discuss the preliminary concepts and theoretical frameworks that form the basis of our proposed methodology. We introduce intuitionistic fuzzy sets (IFS) and we explained the $t$-norm, $t$-conorm, Aczel-Alsina $t$-norm, Aczel-Alsina operations on IFN, which facilitates the aggregation of intuitionistic fuzzy information and linguistic terms. Furthermore, we present the IFS TOPSIS method, which enables the ranking and selection of alternatives based on their similarity to the positive ideal solution and their distance from the negative ideal solution. To find out the criteria weights, the entropy measure is applied in the TOPSIS method, which quantifies the relative importance and contribution of each criterion in the decision-making process.

\vspace{5mm}

In Chapter 4, we present a detailed case study that applies our proposed methodology to the assessment of healthcare waste disposal procedures within the context of intuitionistic fuzzy sets. We apply our method to a real case study for which the research data is taken from A.R.Mishra et al.\cite{7}. We demonstrate the computational procedure of the method. We also analyze and compare the results with existing methods.

\vspace{5mm}

Finally, in Chapter 5, we provide the conclusion of our proposed study. We summarize the key findings and contributions of this dissertation, emphasizing the significance of incorporating intuitionistic fuzzy sets, the Aczel-Alsina operations of IFN, and the entropy weight measure in healthcare waste management decision-making. We discuss the implications of our findings for healthcare facilities, waste management authorities, and policymakers, and highlight potential areas for future research and improvements in the field. 