\chapter{Literature Review}


\label{sec:start}
\begin{flushleft}
    

Healthcare waste disposal is a crucial aspect of ensuring public health and environmental safety. Healthcare waste includes any waste generated by healthcare facilities, medical laboratories, biomedical research facilities, and other sources that may contain infectious, chemical, or radioactive substances. A substantial body of research has focused on evaluating and selecting appropriate healthcare waste disposal methods. In this section, we review the existing literature related to healthcare waste management, decision-making approaches, the use of intuitionistic fuzzy sets, the Aczel-Alsina operations, and entropy measure in decision-making processes.

\vspace{5mm}

A significant challenge in this issue is that in numerous developing
countries, HCW is still disposed of together with other types
of waste, thereby creating critical health risks to the general population,
municipal workers, and the environment.
Numerous studies have highlighted the importance of proper healthcare waste management to prevent the spread of infections and reduce environmental pollution. The World Health Organization (WHO) has emphasized the need for effective waste management practices to ensure public safety and minimize health risks. Research has examined various aspects of healthcare waste management, including waste characterization, collection, storage, transportation, treatment, and disposal. The selection of an optimal disposal method is crucial to mitigate the potential risks associated with healthcare waste.

\vspace{5mm}

Decision-making in healthcare waste management involves considering multiple criteria and stakeholders' preferences. Traditional decision-making methods such as cost-benefit analysis and analytical hierarchy process (AHP) have been widely used in this domain. However, these methods often assume crisp and precise values, which may not accurately capture the uncertainties and imprecise nature of healthcare waste management. To address these limitations, researchers have explored the application of fuzzy sets, which allow for the representation of uncertain and imprecise information. Dursun et al., 2011a \cite{4} introduced a couple of
MCDM methods to perform an investigation based on fuzzy logic
and multi-step hierarchical framework for the best HCW treatment
assessment. Dursun et al., 2011b \cite{3} introduced a Multi-Criteria
Group Decision Making (MCGDM) approach considering the concepts
of fuzzy measure to evaluate the HCW that were available for
Istanbul within the context of FSs, which facilitates the application
of inaccurate data given as linguistic variables to the analyses.
Literature has consisted of various MCDM methods that have
been developed under various uncertain environments.

\vspace{5mm}

Intuitionistic fuzzy sets (IFS) have emerged as a powerful tool for decision-making under uncertainty and ambiguity. Several studies have demonstrated the effectiveness of IFS in handling vagueness and uncertainty in decision-making problems. In the context of healthcare waste management, researchers have employed IFS to model subjective preferences and linguistic terms associated with criteria evaluation and alternative ranking. In last decades, a number of previous studies have used intuitionistic fuzzy
sets theory, and decision making methods in the different application
are such as manufacturing,
sustainable and green supply chain, healthcare waste recycling, green and sustainable human resource. In recent years, several studies have been conducted
into the ways of selecting suitable HCW disposal alternatives hat
have employed the fuzzy set-based linguistic values to represent
experts’ views. 

\vspace{5mm}

The Aczel-Alsina operations provide a framework for combining intuitionistic fuzzy numbers (IFNs) to perform arithmetic operations such as addition, subtraction, multiplication, and division. These operations enable decision-makers to aggregate and manipulate intuitionistic fuzzy information effectively. Researchers have applied the Aczel-Alsina operations in various decision-making problems to handle the uncertainty and imprecision associated with linguistic terms and preferences.
For example,  T. Senapati et al. \cite{10} proposed a MADM
technique in consideration of the IFAAWG (Intuitionistic Fuzzy Aczel-Alsina weighted geometric) operator of IFNs. An IF
MADM approach based on the AA geometric aggregation operators is
proposed to handle IF MADM within IFSs. They demonstrated that the Aczel-Alsina operations effectively captured the linguistic terms and uncertainties in the decision-making process.

\vspace{5mm}

The Technique for Order of Preference by Similarity to Ideal Solution (TOPSIS) is a well-established multi-criteria decision analysis method. Originally developed by Hwang and Yoon (1981)\cite{5},
TOPSIS is a largely used decision-making
process to prioritize difficult problems in decision management.
On the TOPSIS method, alternatives are ranked
from best to worst. It has been extensively used in various fields to rank alternatives based on their similarity to the positive ideal solution and distance from the negative ideal solution. In the domain of healthcare waste management, TOPSIS has been employed to evaluate and select the optimal disposal method considering multiple criteria.The integration of intuitionistic fuzzy sets and TOPSIS has gained attention in decision-making literature.For example. B. D. Rouyendegh et al.\cite{9} applied Intuitionistic Fuzzy TOPSIS method for green supplier selection
problem. The use of intuitionistic fuzzy TOPSIS has been applied in various other domains, including environmental management, transportation, and healthcare.

\vspace{5mm}

The entropy measure is commonly used to calculate the weights of criteria in decision-making processes. It quantifies the level of diversity or uncertainty associated with the criteria values. By utilizing entropy, decision-makers can identify the relative importance and contribution of each criterion in the decision-making process. The entropy measure provides a systematic approach to determine the criteria weights based on the available information. 
N. Alkan and C. Kahraman\cite{1} proposed two different new approaches of q-ROF TOPSIS
method have been proposed to select the best COVID-19 strategy
and rank strategies with their details. One of the approach is q-ROF entropy measure based TOPSIS approach. This is the approach we have considered to evaluate the criteria weights in our proposed TOPSIS method. The researchers demonstrated that the entropy-based approach effectively addressed the imprecise and uncertain nature of the criteria, allowing for a more accurate determination of their relative importance.

\vspace{5mm}

To make an HCW management system with high
sustainability, the system needs to be socially suitable, economically
reasonable, and environmentally effective\cite{8}. As a result, the assessment of HCW disposal options,
which takes into consideration the requirement for trade-off
numerous contradictory criteria with intrinsic inaccuracy and
vagueness, is an MCGDM problem of high importance.

Liu et al. (2013)\cite{6} extended an approach using the VIKOR technique
based on FSs and ordered weighted averaging (OWA) operator to
evaluate health-care waste disposal problem. In that study, four
alternatives as health-care waste treatment such as “landfill”, “microwave”,
“steam sterilization,” and “incineration” are considered. In
another study, researchers pioneered a hybrid fuzzy MCDM
model based on MULTIMOORA(Multiple Objective Optimization on the Basis of Ratio Analysis plus
Full Multiplicative Form) and modified 2-tuple DEMATEL (Decision
Making Trial And Evaluation Laboratory) methods to
select the best health-carewaste process. In that study, to select the
best alternative, criteria are classifies based on four main aspects
including social, environmental, economic as well as technical. 

\vspace{5mm}

A.R. Mishra et al. \cite{7} carried out a survey based on interviews,
questionnaires, and literature reviews in a way to explore the significant
criteria to evaluate the most accurate HCW disposal
alternatives.For the purpose of choosing the most appropriate criteria to
evaluate the best HCW disposal alternatives, it is necessary to
decide on a structure in such a way to manage the assessment
procedure of these sources and then make sure of the fact that the
criteria chosen have the capacity of covering all of the aspects of
this framework. As a result, the researchers a framework
 in a way to demonstrate the critical criteria that
can be considered in choosing the most suitable sources.
The HCW disposal approaches are incineration, steam sterilization,
microwave, and landfill disposal. These methods are evaluated
with respect to given six criteria, which are “release with
health effects,” “treatment effectiveness,” “reliability,” “waste residuals,”
“public acceptance”, and “cost.”

\vspace{5mm}

In reviewing the existing literature, it was observed that the researchers have applied algebraic operators in healthcare waste management decision-making. However, a research gap was identified in the application of Aczel-Alsina operations. While algebraic operators have been utilized to aggregate information and linguistic terms, the specific use of Aczel-Alsina operations, which incorporate intuitionistic fuzzy sets and enhance the representation of decision-makers' preferences, has been largely overlooked.

\vspace{5mm}

While previous studies have explored other multi criteria decision making (MCDM) methods in intuitionistic fuzzy (IF) environment for healthcare waste disposal method selection,  there is a lack of application of the TOPSIS method in an intuitionistic fuzzy (IF) environment for healthcare waste disposal method selection.

\vspace{5mm}

Additionally,while several studies have explored other methods of calculating criteria weights for HCW disposal problems there is a noticeable research gap in the application of the entropy measure to calculate criteria weights within the context of IF-TOPSIS for healthcare waste management decision-making. 

\vspace{5mm}

By addressing these research gaps and incorporating the Aczel-Alsina operations, the TOPSIS method in an intuitionistic fuzzy environment, and the entropy measure for criteria weight calculation, this dissertation aims to contribute to the existing body of knowledge in healthcare waste management decision-making.
\end{flushleft}