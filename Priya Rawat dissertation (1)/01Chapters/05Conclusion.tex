\chapter{Conclusion}
\begin{flushleft}

Healthcare waste poses significant concerns that need to be addressed for the safety of individuals and the environment. By implementing efficient waste management strategies, healthcare facilities can contribute to a healthier and more sustainable environment.

\vspace{3mm}

In conclusion, this dissertation aimed to address the challenge of selecting the optimum healthcare waste disposal method by utilizing the TOPSIS method in an intuitionistic fuzzy (IF) environment, incorporating Aczel-Alsina operations and entropy measure for criteria weight calculation. Through an extensive literature review, research gaps were identified, highlighting the need for the integration of these specific methodologies in the context of healthcare waste management decision-making.

\vspace{3mm}

The research findings demonstrated the effectiveness and applicability of the proposed approach. The AA operations of IFS are noted in this
study, driven by the AA t-NM and t-CM concepts. By considering linguistic terms and incorporating decision-makers' preferences through the utilization of Aczel-Alsina operations, the decision-making process became more comprehensive and realistic.

\vspace{3mm}

The IF-TOPSIS method provided a robust framework for evaluating alternative healthcare waste disposal methods, considering the inherent uncertainties and linguistic expressions associated with decision-making in this domain.Furthermore, the application of the entropy measure for criteria weight calculation enhanced the objectivity and reliability of the decision-making process.

\vspace{3mm}

The developed framework is applied to rank the HCW disposal method options. The outcome demonstrates that 
landfill disposal (A4) should be chosen as the best HCW disposal method. 
The comparison of the rankings obtained using different decision-making approaches emphasized the importance of selecting an appropriate methodology. While variations in rankings were observed, there was consistency in the prioritization of alternatives A2 and A3 across most approaches. These findings highlight the complex nature of healthcare waste management decision-making and the significance of selecting a method that aligns with the specific context and objectives.

\vspace{3mm}

By filling the identified research gaps and contributing a novel decision-making framework, this dissertation advances the field of healthcare waste management. The proposed approach provides decision-makers with a comprehensive and robust tool for selecting the optimum healthcare waste disposal method, considering multiple criteria, linguistic expressions, and uncertainty.

\vspace{3mm}

It is important to note that this dissertation is not without limitations. The effectiveness and generalizability of the proposed approach may vary depending on the specific healthcare waste management context and the availability of data. Further research and validation studies are recommended to evaluate the approach in different scenarios and to refine the methodology based on empirical evidence.

\vspace{3mm}

Overall, this dissertation contributes to the existing body of knowledge by providing a practical and effective framework for healthcare waste management decision-making. The integration of the TOPSIS method in an intuitionistic fuzzy environment, Aczel-Alsina operations, and the entropy measure enhances the decision-making process, facilitating informed and optimal decisions in healthcare waste disposal method selection.


\end{flushleft}