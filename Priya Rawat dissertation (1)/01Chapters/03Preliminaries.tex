\chapter{Preliminaries}
\begin{flushleft}
\section{Intuitionistic Fuzzy Set}
 The fuzzy set theory proposed by Zadeh $(1965)$ \cite{12} deals
with the uncertainty in human decisions. Then, the Intuitionistic Fuzzy Set (IFS)
was introduced by Atanassov $(1986)$ \cite{2}. IFS has been applied
to a lot of various areas, including logic programming,
medical diagnosis, decision-making, evaluation function,
and preference relation. In this section, there are some descriptions of the IFS.
\vspace{0.3cm}

\linebreak

  An intuitionistic fuzzy set (IFS) $A$ in $E$ is defined as an object of the following form:
\begin{equation*}
    A=\{<x,\mu_A (x),\nu_A(x)> \vert x\in E\}
\end{equation*}
where the functions:
\begin{equation*}
    \mu_A : E \rightarrow [0,1]
\end{equation*}
and
\begin{equation*}
    \nu_A : E \rightarrow [0,1]
\end{equation*}
define the degree of membership and the degree of non-membership of the
element $x \in E$, respectively, and for every $x \in E$:
\begin{equation*}
    0\leq \mu_A (x)+\nu_A(x) \leq 1
\end{equation*}

For each IFS $A$ in $E$, we will call 
\begin{equation*}
    \pi_A (x)= 1-\mu_A (x)-\nu_A(x)
\end{equation*}
the degree of non-determinacy(or uncertainty) of the element $x\in E$ to the intuitionistic fuzzy set $A$.
$\pi_A (x)$ is also known as the intuitionistic fuzzy index or hesitation degree of whether $x$ belongs to $A$ or not.


Obviously, each ordinary fuzzy set may be written as
\begin{equation*}
 \{<x,\mu_A (x),1-\mu_A(x)> \vert x\in E\}   
\end{equation*}
Clearly, in this case $\pi_A(x)=0$ for every $x\in E$.

\newpage
%\section{$t$-NM, $t$-CM, AA $t$-NM}

Aczel and Alsina (1982) presented two new operations to the academic
method. These operations are known as the AA t-NM and the
t-CM, and they place a significant emphasis on the importance of
variation with regard to the activities of parameters.

\vspace{3mm}

\section{$t$-norm or $t$-NM}
A function $T: [0,1] \times [0,1] \rightarrow [0,1]$ is called a
$t$-norm if it satisfies the following four conditions\cite{11}:
\item{(1)} $T(1,\kappa)=\kappa$, for all $\kappa$.
\item{(2)} $T(\kappa,\tau) = T(\tau,\kappa)$, for all $\kappa$ and $\tau$.
\item{(3)} $T(\kappa,T(\tau,z)) = T(T(\kappa,\tau),z)$, for all $\kappa$, $\tau$ and z.
\item{(4)} If $\kappa \leq \kappa'$ and $\tau \leq \tau'$, then $T(\kappa,\tau) \leq T(\kappa',\tau')$

\vspace{3mm}

\section{$t$-conorm or $t$-CM}
A function $S: [0,1] \times [0,1] \rightarrow [0,1]$ is called a
$t$-conorm if it satisfies the following four conditions\cite{11}:
\item{(1)} $S(0,\kappa)=\kappa$, for all $\kappa$.
\item{(2)} $S(\kappa,\tau) = S(\tau,\kappa)$, for all $\kappa$ and $\tau$.
\item{(3)} $S(\kappa,S(\tau,z)) = S(S(\kappa,\tau),z)$, for all $\kappa$, $\tau$ and z.
\item{(4)} If  $\kappa \leq \kappa'$ and $\tau \leq \tau'$, then $S(\kappa,\tau) \leq S(\kappa',\tau')$

\vspace{3mm}

It should be noted that these two classes are dual, i.e., for every $t$-
NM $T$ , the function $S:[0, 1]^2 \rightarrow [0, 1]$ given by $S(\kappa,\tau)=1-T(1-\kappa, 1-\tau)$
is a $t$-CM (sometimes referred to as a $t$-CM dual to $T$ ), and for every $t$-
CM $S$, the function $T : [0, 1]^2 \rightarrow [0, 1]$ given by $T(\kappa,\tau)=1-S(1-\kappa, 1-\tau)$
is a $t$-NM ($t$-NM dual to $S$).

\vspace{3mm}

\section{Aczel and Alsina(AA) $t$-NM and $t$-CM} 
Aczel and Alsina presented two new operations to the academic
method. These operations are known as the AA $t$-NM and the
$t$-CM, and they place a significant emphasis on the importance of
variation with regard to the activities of parameters.
The associated $t$-NMs are designated as $T_A ^\eta$
and are referred to as
(strict) AA $t$-NMs\cite{10}. They can be represented as
\begin{equation*}
    T_A^\eta (\kappa, \tau)= 
    \begin{cases}
    T_D (\kappa, \tau), & \text{if } \eta=0\\
    min (\kappa, \tau), & \text{if } \eta=\infty\\
    \exp^{-((-ln \kappa)^ \eta + (-ln \tau)^ \eta) ^{1/\eta}}, & \text{otherwise}
    \end{cases}
\end{equation*}

Note that when we include the extremal $t$-NMs, we acquire an associated
AA class ($ T_A^\eta$), $\eta \in [0, \infty]$ of $t$-NMs that consists of $t$-NMs that are
continuous and strictly increasing in parameter $\eta$.
By way of illustration, dual $t$-CMs $ S_A^\eta$
to strict AA $t$-noms $ T_A^\eta$
are
produced by ‘‘additive generators’’ $s_n^\eta=(-log(1-\kappa))^\eta$, and are
represented by
\begin{equation*}
    S_A^\eta (\kappa, \tau)= 
    \begin{cases}
    S_D (\kappa, \tau), & \text{if } \eta=0\\
    max (\kappa, \tau), & \text{if } \eta=\infty\\
    1-\exp^{-((-ln \kappa)^ \eta + (-ln \tau)^ \eta) ^{1/\eta}}, & \text{otherwise}
    \end{cases}
\end{equation*}

Note that when we include the extremal $t$-CMs, we acquire an associated
AA class ($ S_A^\eta$), $\eta \in [0, \infty]$ of $t$-CMs that consists of $t$-CMs that are
continuous and strictly decreasing in parameter $\eta$.


\section{Aczel and Alsina(AA) operations on IFNs}


Suppose, $\phi$ and $\psi$ are
both IFSs. Then the generalized intersection and the generalized union
are calculated by using the formula:

\vspace{3mm}

$\phi \cap_{T,S} \psi =\{< \kappa, T\{\mu_{\phi}(\kappa), \mu_{\psi}(\kappa)\}, S\{\nu_{\phi}(\kappa), \nu_{\psi}(\kappa)\}>| \kappa \in E \}$

$\phi \cup_{S,T} \psi =\{< \kappa, S\{\mu_{\phi}(\kappa), \mu_{\psi}(\kappa)\}, T\{\nu_{\phi}(\kappa), \nu_{\psi}(\kappa)\}>| \kappa \in E \}$

\vspace{3mm}

in which $T$ refers a $t$-NM and $S$ refers a $t$-CM.

\vspace{3mm}

Let t-NM, $T$ and t-CM, $S$ be the AA product $T_A$ and AA sum $S_A$ respectively.
Then motivated by the above two equations, the generalized intersection and
union over two IFSs $\phi$ and $\psi$ turn out to be the AA product (shown
by $\phi \otimes \psi$) and AA sum (shown by $\phi \oplus \psi$) over two IFSs $\phi$ and $\psi$
respectively, in the following way \cite{10}:
\vspace{3mm}

$\phi \otimes_{AA} \psi =\{< \kappa, T_A\{\mu_{\phi}(\kappa), \mu_{\psi}(\kappa)\}, S_A\{\nu_{\phi}(\kappa), \nu_{\psi}(\kappa)\}>| \kappa \in E \}$

$\phi \oplus_{AA} \psi =\{< \kappa, S_A\{\mu_{\phi}(\kappa), \mu_{\psi}(\kappa)\}, T_A\{\nu_{\phi}(\kappa), \nu_{\psi}(\kappa)\}>| \kappa \in E \}$


\vspace{3mm}

\textbf{Definition:} Let  $A=\{<x,\mu_A (x),\nu_A(x)> \vert x\in E\}$,

\vspace{3mm}

$A_1=\{<x,\mu_{A_1} (x),\nu_{A_1} (x)> \vert x\in E\}$ 

\vspace{3mm}

$\text{and} A_2=\{<x,\mu_{A_2} (x),\nu_{A_2} (x)> \vert x\in E\}$ 

\vspace{3mm}

be three IFNs, $\tau \geq 1$, and $\eta > 0$. Following that, the AA t-NM and t-CM operations of IFNs are summarized as follows:

\vspace{3mm}

%\begin{enumerate}
    \item{(i)} $A_1 \oplus_{AA} A_2 = \bigg < 1- e^{-((-\ln{(1-\mu_{A_1})})^ {\tau}+(-\ln{(1-\mu_{A_2})})^ {\tau})^{1/\tau}}, e^{-((-\ln{(\nu_{A_1})})^ {\tau}+(-\ln{(\nu_{A_2})})^ {\tau})^{1/\tau}} \bigg>$
    
   \item{(ii)} $A_1 \otimes_{AA} A_2 = \bigg < e^{-((-\ln{(\mu_{A_1})})^ {\tau}+(-\ln{(\mu_{A_2})})^ {\tau})^{1/\tau}} , 1- e^{-((-\ln{(1-\nu_{A_1})})^ {\tau}+(-\ln{(1-\nu_{A_2})})^ {\tau})^{1/\tau}} \bigg>$

   \item{(iii)} $\eta \bigtimes_{AA} A =\bigg< 1- e^{-(\eta(-\ln{(1-\mu_{A})})^ {\tau})^{1/\tau}}, e^{-(\eta(-\ln{(\nu_{A})})^ {\tau})^{1/\tau}} \bigg >$

   \item{(iv)} A^{\big{\eta_{AA}}}= \bigg< e^{-(\eta(-\ln{(\mu_{A})})^ {\tau})^{1/\tau}}, 1- e^{-(\eta(-\ln{(1-\nu_{A})})^ {\tau})^{1/\tau}} \bigg>
%\end{enumerate}
\newline
 \section{Entropy based IF-TOPSIS method with Aczel and Alsina operations}

 \vspace{3mm}
 
TOPSIS, which was developed by Hwang and Yoon (1981)\cite{5}, is based on the idea that
the chosen alternative should have the shortest distance from the positive ideal solution
and the longest distance from the negative ideal solution. TOPSIS simultaneously considers the distances to both ideal solution
and negative ideal solution, and a preference order is ranked according to their relative
closeness, and a combination of these two distance measures. The formulas for the calculation of criteria weights are constructed based on the
entropy of q-ROFSs developed by Liu et al. \cite{13}

\vspace{3mm}

The Intuitionistic Fuzzy TOPSIS model was introduced by Rouyendegh\cite{9} for the
evaluation of the alternatives.

The algorithm consists of nine steps as follows:
\newlist{steps}{enumerate}{1}
\setlist[steps, 1]{label = Step \arabic*:}
\begin{steps}
    

\item  Originate the alternative and criterion sets

In the MCDM procedure, the objective is to choose the optimal
option from amongst the m alternatives $A={A_1,A_2,...,A_m}$ under
the criterion set $C={C_1,C_2,...,C_n}$. Consider a committee of l experts
 $L={l_1,l_2,...,l_l}$ has been formed to obtain the desirable
alternative(s).
\item Determine the weights of the importance of DMs
The importance of each decision-maker is considered as linguistic terms articulated in IFNs.Let $Dl=[\mu_l,\nu_l, \pi_l]$ is the IFN for $l^{th}$ DM ranking. The importance of $l_th$ DM is evaluated with the following formula:
\begin{equation}
    \lambda_l=\frac{[\mu_l+\pi_l(\frac{\mu_l}{\mu_l +\nu_l})]}{\sum_{l=1}^{k} [\mu_l+\pi_l(\frac{\mu_l}{\mu_l +\nu_l})]}
\end{equation}
\begin{equation*}
    \lambda_l \in [0,1] 
    and
 \sum_{l=1}^{k} \lambda_l =1
\end{equation*}
\item Create IF-aggregated decision matrix (IF-ADM)

It is assumed that $\tau =1$. For four HCW disposal
alternative methods, the IF AA operators are used to
obtain an overall alternative values. To achieve a result, each individual opinion
obtained from a group of DMs should be put together into a
single opinion to construct the IF-ADM model.
\begin{equation*}
    R^{(l)}=(r_{ij}^{(l)})_{m*n} \text{is the IFDM of each DM}
\end{equation*}
\begin{equation*}
    \lambda=\{\lambda_1,\lambda_2,\lambda_3,...,\lambda_k\} \text{is the weight of the decision maker}
\end{equation*}
\begin{equation*}
    R=(r_{ij})_{m'\times n'} 
\end{equation*}
\begin{equation*}
    r_{ij}=IFAAWA r_\lambda (r_{ij}^{(1)},r_{ij}^{(2)},...,r_{ij}^{(k)})=\lambda_1 r_{ij}^{(1)}+\lambda_2 r_{ij}^{(2)}+...+\lambda_k r_{ij}^{(k)}
\end{equation*}
\begin{equation}
    r_{ij}= (\lambda_1 \times_{AA} r_{ij}^{(1)}) \oplus_{AA} (\lambda_2 \times_{AA} r_{ij}^{(2)}) \oplus_{AA} ... \oplus_{AA} (\lambda_k \times_{AA} r_{ij}^{(k)})
\end{equation}
%\begin{equation*}
   % =\bigg[1-\prod_{l=1}^k (1-\mu_{ij}^{(l)})^{\lambda l}, \prod_{l=1}^k (\nu_{ij}^{(l)})^{\lambda l}, \prod_{l=1}^k (1-\mu_{ij}^{(l)})^{\lambda l}- \prod_{l=1}^k (1-\nu_{ij}^{(l)})^{\lambda l} \bigg]
%\end{equation*}





\item Obtain the entropy values\cite{1} of each IFN in the
aggregated decision matrix by using the following:
\begin{equation}
    KE_i_j(x)=\frac{1}{\sqrt{2}} \sqrt{(\mu(x))^2 +(\nu(x))^2+(\mu(x)+\nu(x))^2}
\end{equation}
\begin{equation*}
    EN_i_j(x)=1-KE_i_j(x)
\end{equation*}
\begin{equation}
    EN_i_j(x) =1-\frac{1}{\sqrt{2}} \sqrt{(\mu(x))^2 +(\nu(x))^2+(\mu(x)+\nu(x))^2}
\end{equation}

\item Calculate the weights of the criteria. 

The weights of the criteria
are calculated based on entropy values by using the following:
\begin{equation}
    w_j=\frac{1-\xi_j}{\sum_{j=1}^{n}(1-\xi_j)},
    j=1,2,...,n
\end{equation}
where
\begin{equation}
     \xi_j=\frac{\sum_{i=1}^{m} EN_i_j}{\sum_{i=1}^{m}\sum_{j=1}^{n} EN_i_j},
\end{equation}
indicates the intuitionistic fuzzy entropy value.



\item Determine the intuitionistic fuzzy positive and
negative-ideal solutions

J1 is the benefit criteria and J2 is cost criteria. $A^+$ is the
intuitionistic fuzzy positive-ideal solution and $A^-$ is the
intuitionistic fuzzy negative-ideal solution. Then $A^+$ and
$A^-$ are obtained as follows:
\begin{equation}
    A^+= (r_1^{'*},r_2^{'*},...,r_n^{'*}), r_j^{'*}=(\mu_j^{'*},\nu_j{'*},\pi_j{'*}), j= 1,2,...,n
\end{equation}
\begin{equation}
    A^-= (r_1^{'-},r_2^{'-},...,r_n^{'-}), r_j^{'-}=(\mu_j^{'-},\nu_j{'-},\pi_j{'-}), j= 1,2,...,n
\end{equation}
where
\begin{equation*}
    \mu_j^{'*}=\{(\max_{i} \{\mu_i_j^{'}\}\vert j\in J_1) , (\min_{i} \{\mu_i_j^{'}\}\vert j\in J_2)\}
\end{equation*}
\begin{equation*}
    \nu_j^{'*}=\{(\min_{i} \{\nu_i_j^{'}\}\vert j\in J_1) , (\max_{i} \{\nu_i_j^{'}\}\vert j\in J_2)\}
\end{equation*}
\begin{equation*}
    \pi_j^{'*}=\{(1-\max_{i} \{\mu_i_j^{'}\}-\min_{i} \{\nu_i_j^{'}\}\vert j\in J_1) , (1-\min_{i} \{\mu_i_j^{'}\}-\max_{i} \{\nu_i_j^{'}\}\vert j\in J_2)\}
\end{equation*}
\begin{equation*}
    \mu_j^{'-}=\{(\min_{i} \{\mu_i_j^{'}\}\vert j\in J_1) , (\max_{i} \{\mu_i_j^{'}\}\vert j\in J_2)\}
\end{equation*}
\begin{equation*}
    \nu_j^{'-}=\{(\max_{i} \{\nu_i_j^{'}\}\vert j\in J_1) , (\min_{i} \{\nu_i_j^{'}\}\vert j\in J_2)\}
\end{equation*}
\begin{equation*}
    \pi_j^{'-}=\{(1-\min_{i} \{\mu_i_j^{'}\}-\max_{i} \{\nu_i_j^{'}\}\vert j\in J_1) , (1-\max_{i} \{\mu_i_j^{'}\}-\min_{i} \{\nu_i_j^{'}\}\vert j\in J_2)\}
\end{equation*}



\item Determine the separation measures between the
alternatives

Determine separation measures by calculating the
weighted distances for each alternative according to positive ideal solutions
$A^+$ and negative-ideal solutions $A^-$. Distance between
each alternative, and positive-ideal solution $A^+$ as well as negative-ideal solution $A^+$  are calculated by utilizing Euclidean distance. After selecting the distance measure, the separation measures are calculated. The separation measures of an alternative from the positive and negative-ideal solutions are indicated by $S_i^+$ and $S_i^-$.

\begin{equation}
    S_i^* = \sqrt{\frac{1}{2} \sum_{j=1}^{n} w_j [(\mu_i_j^{'}-\mu_j^{'*})^2 + (\nu_i_j^{'}-\nu_j^{'*})^2 +(\pi_i_j^{'}-\pi_j^{'*})^2]}
\end{equation}
\begin{equation}
    S_i^- = \sqrt{\frac{1}{2} \sum_{j=1}^{n} w_j [(\mu_i_j^{'}-\mu_j^{'-})^2 + (\nu_i_j^{'}-\nu_j^{'-})^2 +(\pi_i_j^{'}-\pi_j^{'-})^2]}
\end{equation}
\item Calculate the relative closeness coefficient of alternative $A_i$ 

The relative closeness coefficient of an alternative $A_i$
to the intuitionistic fuzzy positive-ideal solution
$A^+$ is defined as follows:
\begin{equation}
   C_i^*=\frac{S_i^-}{S_i^+ + S_i^-}
\end{equation}
and
\begin{equation*}
    0\leq C_i^*\leq 1
\end{equation*}

\item Determine the optimal alternative(s)   

After determining the relative closeness coefficient,the $C_i^*$s are sorted in descending order to achieve the order of preference.
\end{steps}
\end{flushleft}

